\usetikzlibrary{patterns}
\begin{wrapfigure}{r}{0.35\textwidth}
\caption{Зона устойчивости явного метода}\label{wrap-fig:1}
\centering
\begin{tikzpicture}[yscale=1.2,xscale=1.2]
\datavisualization[school book axes={standard labels},
    visualize as smooth line,
    clean ticks,
    x axis= {label = $Re(\Delta t\cdot \lambda)$,unit length=8mm, ticks={step=1}},
    y axis= {label = $Im(\Delta t\cdot \lambda)$, unit length=8mm, ticks={step=1}},
    visualize as smooth line/.list={1,2},
    visualize as circle/.list={a},
    visualize as line/.list={3,...,14},
    3 = {style={->}},
    4 = {style={->}},
    5 = {style={->}},
    6 = {style={->}},
    7 = {style={->}},
    8 = {style={->}},
    9 = {style={->}},
    10 = {style={->}},
    11 = {style={->}},
    12 = {style={->}},
    13 = {style={->}},
    14 = {style={->}}
    %b = {style={->}}
    ]
data [format=function] {
    var set : {1};
    var t : interval [0:2*pi];
    func x = \value{set} * cos(\value t r) - 1;
    func y = \value{set} * sin(\value t r);
}
;
\begin{scope}[on background layer, yscale=0.8,xscale=0.8]
 \fill[black,opacity=.2,even odd rule]
 (-1,0) circle[radius=1];
\end{scope} 
\end{tikzpicture}
\end{wrapfigure}